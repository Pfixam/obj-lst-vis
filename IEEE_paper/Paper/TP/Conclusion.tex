\section{Conclusions}
	 Nowadays, urban driving simulation software has a major impact on the development process of new autonomous driving functions. Due to the constant further development of the range of functions, the adaptable \ac{API}, \ac{ROS} integration and as well the rapidly growing community, open-source simulators like CARLA offer excellent opportunities for testing and evaluating individual sensor suites and functions. Freely adaptable scenarios and weather conditions offer the developer a lot of flexibility when creating their own test cases.\\

	 In this specific scenario a real-time object detection system combined with an object tracker are used.
	 As said in \cref{C}, the data processing time is highly dependent on the hardware on which the calculation runs. The delay mentioned in \cref{Results} between the time of recording and visualization leads to outdated results in \ac{RVIZ}. This is the most urgent problem to be solved. Further improvements can be achieved with training the \ac{YOLO} algorithm.
	 Synchronizing the published object lists data by considering the time stamps of the sensor image frames itself and not of the published \ac{ROS} data flow should also lead to more comparable data.
	 However, this only will be useful if the acquisition time by \ac{YOLO} and the code itself will improve.
%	 Synchronizing the published objects lists data by considering the time stamps of the calculated data as well as the \ac{GT} data should also lead to more comparable data. However, this only will be useful if the acquisition time by \ac{YOLO} and the code itself will improve.
	 
	 
	
	 
	 
	
	 
	
	 
	 
	 
	